% JuliaCon proceedings template
%\documentclass[]{article}
\documentclass{juliacon}
\setcounter{page}{1}

%\usepackage[utf8]{inputenc}

\usepackage{graphicx}
\usepackage[left=1.00in, right=1.00in, top=1.00in, bottom=1.00in]{geometry}

\usepackage{dirtytalk}
\usepackage[normalem]{ulem}
\usepackage{tikz-cd}
\usepackage{units}
\usepackage{algorithm}
\usepackage{algpseudocode}
\usepackage{alltt}
\usepackage{mathrsfs}
\usepackage{amssymb}
\usepackage{amsmath}
\DeclareMathOperator\cis{cis}
\newenvironment{rcases}
  {\left.\begin{aligned}}
  {\end{aligned}\right\rbrace}
\newenvironment{acases}
  {\left.\begin{aligned}}
  {\end{aligned}\right}

% (font shortcuts)
\usepackage{amsfonts}
\newcommand{\mb}[1]{\mathbb{#1}}
\newcommand{\mc}[1]{\mathcal{#1}}
\newcommand{\ms}[1]{\mathscr{#1}}
\newcommand{\mf}[1]{\frak{#1}}

% (arrow shortcuts)
\newcommand{\ra}{\rightarrow}
\newcommand{\lra}{\longrightarrow}
\newcommand{\la}{\leftarrow}
\newcommand{\lla}{\longleftarrow}
\newcommand{\Ra}{\Rightarrow}
\newcommand{\Lra}{\Longrightarrow}
\newcommand{\La}{\Leftarrow}
\newcommand{\Lla}{\Longleftarrow}
\newcommand{\lr}{\leftrightarrow}
\newcommand{\llr}{\longleftrightarrow}
\newcommand{\Lr}{\Leftrightarrow}
\newcommand{\Llr}{\Longleftrightarrow}
\newcommand{\ua}{\uparrow}
\newcommand{\da}{\downarrow}
\newcommand{\Ua}{\Uparrow}
\newcommand{\Da}{\Downarrow}

% (match parenthesis)
\newcommand{\mlr}[1]{\left|#1\right|}
\newcommand{\plr}[1]{\left(#1\right)}
\newcommand{\blr}[1]{\left[#1\right]}
\newcommand{\nlr}[1]{\mlr{\mlr{#1}}}
\newcommand{\nlrp}[2]{\nlr{#1}_{#2}}

% (exponent shortcuts)
\newcommand{\inv}{^{-1}}
\newcommand{\nrt}[2]{\sqrt[\leftroot{-2}\uproot{2}#1]{#2}}

% (annotation shortcuts)
\newcommand{\conj}[1]{\overline{#1}}
\newcommand{\ol}[1]{\overline{#1}}
\newcommand{\ul}[1]{\underline{#1}}
\newcommand{\os}[2]{\overset{#1}{#2}}
\newcommand{\us}[2]{\underset{#1}{#2}}
\newcommand{\ob}[2]{\overbrace{#2}^{#1}}
\newcommand{\ub}[2]{\underbrace{#2}_{#1}}
\newcommand{\bs}{\backslash}
\newcommand{\ds}{\displaystyle}

% (set builder)
\newcommand{\set}[1]{\left\{ #1 \right\}}
\newcommand{\setc}[2]{\left\{ #1 : #2 \right\}}
\newcommand{\setm}[2]{\left\{ #1 \, \middle| \, #2 \right\}}
\newcommand{\Cp}[2]{C^{#1}(#2)}
\newcommand{\Cb}[3]{C^{#1}{[}#2,#3{]}}
\newcommand{\Cpb}[3]{C^{#1}({[}#2,#3{]})}
%\newcommand{\C}[3]{\Cb{#1}{#2}{#3}}

% (group generator)
\newcommand{\gen}[1]{\left\langle #1 \right\rangle}

% (functions)
\newcommand{\im}[1]{\text{im}(#1)}
\newcommand{\range}[1]{\text{range}(#1)}
\newcommand{\domain}[1]{\text{domain}(#1)}
\newcommand{\dist}[1]{(#1)}
\newcommand{\sgn}{\text{sgn}}

% (sums)
\newcommand{\lset}[2]{#2_1,\dots,#2_{#1}}
\newcommand{\lsum}[2]{#2_1+\dots+#2_{#1}}
\newcommand{\norm}[3]{|#2_1|^{#1}+\dots+|#2_{#3}|^{#1}}
\newcommand{\norms}[4]{\sum_{#2=1}^{#4}|#3_{#2}|^{#1}}
\newcommand{\lc}[3]{#2_1#3_1+\dots+#2_{#1}#3_{#1}}
\newcommand{\lcs}[4]{\sum_{#4=1}^{#1}#2_{#4}#3_{#4}}

% (Linear Algebra)
\newcommand{\mat}[1]{\begin{bmatrix}#1\end{bmatrix}}
\newcommand{\pmat}[1]{\begin{pmatrix}#1\end{pmatrix}}
%\newcommand{\dim}[1]{\text{dim}(#1)}
\newcommand{\rnk}[1]{\text{rank}(#1)}
\newcommand{\nul}[1]{\text{nul}(#1)}
\newcommand{\spn}[1]{\text{span}\,#1}
\newcommand{\col}[1]{\text{col}(#1)}
%\newcommand{\ker}[1]{\text{ker}(#1)}
\newcommand{\row}[1]{\text{row}(#1)}
\newcommand{\area}[1]{\text{area}(#1)}
\newcommand{\nullity}[1]{\text{nullity}(#1)}
\newcommand{\proj}[2]{\text{proj}_{#1}\left(#2\right)}
\newcommand{\diam}[1]{\text{diam}\,#1}

% (Vectors common)
\newcommand{\myvec}[1]{\vec{#1}}
\newcommand{\va}{\myvec{a}}
\newcommand{\vb}{\myvec{b}}
\newcommand{\vc}{\myvec{c}}
\newcommand{\vd}{\myvec{d}}
\newcommand{\ve}{\myvec{e}}
\newcommand{\vf}{\myvec{f}}
\newcommand{\vg}{\myvec{g}}
\newcommand{\vh}{\myvec{h}}
\newcommand{\vi}{\myvec{i}}
\newcommand{\vj}{\myvec{j}}
\newcommand{\vk}{\myvec{k}}
\newcommand{\vl}{\myvec{l}}
\newcommand{\vm}{\myvec{m}}
\newcommand{\vn}{\myvec{n}}
\newcommand{\vo}{\myvec{o}}
\newcommand{\vp}{\myvec{p}}
\newcommand{\vq}{\myvec{q}}
\newcommand{\vr}{\myvec{r}}
\newcommand{\vs}{\myvec{s}}
\newcommand{\vt}{\myvec{t}}
\newcommand{\vu}{\myvec{u}}
\newcommand{\vv}{\myvec{v}}
\newcommand{\vw}{\myvec{w}}
\newcommand{\vx}{\myvec{x}}
\newcommand{\vy}{\myvec{y}}
\newcommand{\vz}{\myvec{z}}
\newcommand{\vzero}{\myvec{0}}


% Theorems and Propositions
%\usepackage{amsthm}
%\newtheorem{theorem}{Theorem}
%\newtheorem{proposition}{Proposition}

%\theoremstyle{definition}
%\newtheorem{definition}{Definition}

%\theoremstyle{remark}
%\newtheorem*{remark}{Remark}
%\newtheorem{example}{Example}
%\newtheorem*{recall}{Recall}
%\newtheorem*{note}{Note}
%\newtheorem*{observe}{Observe}
%\newtheorem*{question}{\underline{Question}}
%\newtheorem*{fact}{Fact}
%\newtheorem{corollary}{Corollary}
%\newtheorem*{lemma}{Lemma}
%\newtheorem{xca}{Exercise}

\usepackage{unicode-math}
\setmonofont[Scale=MatchLowercase]{DejaVu Sans Mono}
\usepackage{scalerel}
\DeclareMathOperator*{\bigboxtimes}{\scalerel*{\boxtimes}{\bigodot}}
\DeclareMathOperator*{\bigominus}{\scalerel*{\ominus}{\bigodot}}
\DeclareMathOperator*{\obackslash}{\text{\reflectbox{$\oslash$}}}

\author{Michael Reed (Crucial Flow Research)}
\title{Differential geometric algebra using Leibniz, Grassmann}
\date{July 25th, 2019}

\begin{document}

%% **************GENERATED FILE, DO NOT EDIT**************

\title{Differential geometric algebra using Leibniz, Grassmann}

\author[1]{Michael Reed}
\affil[1]{Crucial Flow Research, Computational Meta-Linguist}

\keywords{Julia, Geometric Algebra, Differential Geometry, Multiple Dispatch, Metaprogramming}



\maketitle

\begin{abstract}
	The \textit{Grassmann.jl}
	package provides tools for computations based on multi-linear algebra and spin groups using the extended geometric algebra known as Grassmann-Clifford-Hestenes-Leibniz algebra.
	The primary operations are
	outer, regressive, inner, geometric, and cross products along with the Hodge star, adjoint, and multivector reversal operations. 
	The kernelized operations are built on composite sparse tensor products and Hodge duality, with high dimensional support for up to 62 indices and staged caching / precompilation. Code generation enables making concise and extensible definitions.
	The \textit{DirectSum.jl}
	multivector parametric type polymorphism is based on various tangent vector spaces and conformal projective geometry.
	Additionally, interoperability between different sub-algebras is enabled by \textit{AbstractTensors.jl},
	on which the type system is built.
	%\headingtable
\end{abstract}

In this paper, the mathematical foundations and some of the nuances in the definitions specific to the \textit{Grassmann.jl} implementation are concisely described.
The \textit{Grassmann.jl} package and its accompanying support packages provide an extensible platform for computing with geometric algebra at high dimensions. 
The design is based on the \verb`TensorAlgebra` abstract type system interoperability from \textit{AbstractTensors.jl} with a \verb`VectorBundle` type parameter from \textit{DirectSum.jl}. Abstract vector space type operations happen at compile-time, resulting in a differential conformal geometric algebra of hyper-dual multivector forms.

\begin{itemize}
	\item \textbf{DirectSum.jl}: Abstract tangent bundle vector space types (unions, intersections, sums, etc.)
	\item \textbf{AbstractTensors.jl}: Tensor algebra abstract type interoperability with vector bundle parameter	
	\item \textbf{Grassmann.jl}: $\langle$Grassmann-Clifford-Hestenes-Leibniz$\rangle$ differential conformal geometric algebra
	\item \textbf{Leibniz.jl}: Operator algebras for mixed-symmetry multivariate differentiable tensor fields
	\item \textbf{Reduce.jl}: Symbolic parser generator for Julia expressions using REDUCE algebra term rewriter
\end{itemize}

The nature of the geometric algebra code generation enables one to easily extend the abstract product operations to any specific number field type (including Leibniz differentials with \textit{Leibniz.jl} or symbolic coefficients with \textit{Reduce.jl}), by taking advantage of Julia's type system. With the type system, it is possible to construct mixed tensor products with their coefficients, e.g. bivector elements of Lie groups \cite{hestenes}\cite{shahshahani}.

\section{Direct sum parametric type polymorphism}

The \textit{DirectSum.jl} package is a work in progress providing the necessary tools to work with arbitrary dual \verb`VectorBundle` elements with optional origin. Due to the parametric type system for the generating \verb`VectorBundle`, the Julia compiler can fully preallocate and often cache values efficiently ahead of run-time.
Although intended for use with the \textit{Grassmann.jl} package, \verb`DirectSum` can be used independently.

\begin{definition}[Vector bundle of submanifold]
	Let $T^\mu V\in\text{Vect}_{\mb K}$ be a \verb`VectorBundle<:Manifold` of rank $n$,
	\begin{align*}
		T^\mu V &= (n,\mb P,g,\mu), & n,\mu &\in\mb N, & \mb P &\subseteq\gen{v_\infty,v_\emptyset}, & g &:V\times V\ra\mb K
	\end{align*}
	The type \verb+VectorBundle{n,+$\mb P$\verb+,g,+$\mu$\verb+}+ uses \textit{byte-encoded} data available at pre-compilation, where
	$\mb P$ specifies the null-basis from the projective split,
	$g$ is a bilinear form that specifies the metric of the space,
	and $\mu$ is an integer specifying the order of the tangent bundle (i.e. multiplicity limit of Leibniz-Taylor monomials).

	The dual space functor $(\cdot)':\text{Vect}_{\mb K}^\text{op}\ra\text{Vect}_{\mb K}$ 
	is an involution which toggles a dual vector space with inverted signature with property $V' = \text{Hom}(V,\mb K)$ and having the \verb`Basis` generators
	$$\gen{v_1,\dots,v_n,\partial_1,\dots,\partial_n}=T^\mu V\leftrightarrow T^\mu V' = \gen{w_1,\dots,w_n,\epsilon_1,\dots,\epsilon_n}.$$

	The metric signature of the \verb+Basis{V,1}+ elements of a vector space \verb+V+ can be specified with the \verb+V"..."+ constructor by using \verb-+- and \verb+-+ to specify whether the \verb+Basis{V,1}+ element of the corresponding index squares to \verb`+1` or \verb`-1`.
	For example, \verb`V"+++"` constructs a positive definite 3-dimensional \verb`VectorBundle`.
	It is possible to specify an arbitrary \verb`DiagonalForm` for the basis elements with \verb`V"0,0,0"` or \verb`V"-1,1,1,1"`, although the $\pm$ format has better performance.
	Further development of the \textit{DirectSum.jl} will result in more metric types.

	The direct sum operator \verb`⊕` can be used to join spaces (alternatively \verb`+`).
	The direct sum of a \verb`VectorBundle` and its dual \verb`V⊕V'` represents the full mother space \verb`V*`.
	Additionally to the direct-sum operation, several other operations are supported, such as $\cup,\cap,\subseteq,\supseteq$ for set operations.
	Due to the design of the \verb`VectorBundle` type system, these operations make code optimization easier at compile-time by evaluating the bit parameters.
	$$ \bigcup T^{\mu_i}V_i = \plr{|\mb P|+\max\set{n_i-|\mb P_i|}_i,\, \bigcup \mb P_i,\, \cup g_i,\, \max\set{\mu_i}_i} $$
	$$ \bigoplus T^{\mu_i}V_i = \plr{\mlr{\mb P}+\sum (n_i-|\mb P_i|),\, \bigcup \mb P_i,\, \oplus_i g_i,\,\max\set{\mu_i}_i} $$
\end{definition}
\begin{remark}
	Although some of the type operations like $\bigcup$ and $\bigoplus$ are similar and sometimes result in equal values, the union and sum are entirely different operations in general. The \verb`tangent` map is used for specifying $\mu$.
\end{remark}

For any \verb+SubManifold{m,V,s} where+ $\set{n\geq m\in\mb N,V\in\text{Vect}_{\mb K},s\in V}$ we can define the orthogonal space $Z$:
$$T^eV \subset T^\mu W \iff \exists Z\in\text{Vect}_{\mb K}(T^e(V\oplus Z) = T^{e\leq \mu}W,\,V\perp Z),$$
such that computing unions of sub-manifolds is done by inspecting the parameter $s\in V\subseteq W$ and $s\notin Z$.

\section{Tensor basis equivalence classes of Leibniz and Grassmann}

The \verb`AbstractTensors` package is intended for universal interoperability of the abstract \verb`TensorAlgebra` type system.
All \verb`TensorAlgebra{V}` subtypes contain \verb`V` in their type parameters, used to store a \verb`VectorBundle` value obtained from the \textit{DirectSum.jl} package.
By itself, this package does not impose any structure or specifications on the \verb`TensorAlgebra{V}` subtypes and elements, aside from requiring $V$ to be a \verb`VectorBundle`.
This means that different packages can create tensor types with a common underlying \verb`VectorBundle` structure.

\begin{definition}
	Let $V\in\text{Vect}_{\mb k}$ be a \verb`VectorBundle` with dual space $V'$ and the basis elements $w_k:V\ra\mb K$, then for all $x\in V,c\in\mb K$ the properties $(w_i+w_j)(x) = w_i(x)+w_j(x)$ and $(cw_k)(x) = cw_k(x)$ hold.
	An element of a mixed-symmetry \verb`TensorAlgebra{V}` is a multilinear mapping formally constructed by taking the tensor products of linear and multilinear maps,
	$(\bigotimes_k \omega_k)(v_1,\dots,v_{\sum_k p_k}) = \prod_k \omega_k(v_1,\dots,v_{p_k})$.
\end{definition}
\begin{definition}[Mixed-symmetry basis]
	Combining linear basis generating elements with each other using the multi-linear tensor product yields a graded (decomposable) tensor \verb`Basis` $\langle w_{p_1}\otimes\dots\otimes w_{p_k}\rangle_k$, where the \verb`grade` $k$ is determined by the number of basis elements in the tensor product decomposition.
	The algebra is partitioned into symmetric and anti-symmetric tensor equivalence classes.
	For any pair of tensors, either
	\begin{align*}
		&\us{\text{anti-symmetric}}{\omega\otimes\eta = -\eta\otimes\omega}, & &\us{\text{symmetric}}{\omega\otimes\eta = \eta\otimes\omega}.
	\end{align*}
	Typically the $k$ in $\plr{\partial_{p_1}\otimes\cdots\otimes\partial_{p_k}}^{(k)}$ is referred to as the \verb`order` of the element for fully symmetric tensors, which is tracked separately from the \verb`grade` such that $\partial_k\gen{w_j}_r = \gen{\partial_kw_j}_r$ and $(\partial_k)^{(r)}\omega_j = (\partial_kw_j)^{(r)}$.
	Hence, there is a partitioning into \verb`even` grade components $\omega_+$ and \verb`odd` grade components $\omega_-$ with $\omega_++\omega_-=\omega$.
\end{definition}
\begin{remark}
	Observe that the anti-symmetric property implies that $\omega\otimes\omega=0$, while the symmetric property neither implies nor denies such a property.
	In 1862, Grassmann remarked \cite{grassmann-2} that the symmetric algebra of functions is by far more complicated than his anti-symmetric exterior algebra.
	The first part of the book focused on anti-symmetric exterior algebra, while the more complex symmetric function algebra of Leibniz was subject of the second part of the book.
	Elements $\omega_k$ in the space $\Lambda V$ of anti-symmetric algebra are often studied as unit quantum state vectors in a unitary probability space, where $\sum_k\omega_k\neq\bigotimes_k\omega_k$ is entanglement.
\end{remark}
\begin{definition}
	The Grassmann anti-symmetric exterior basis is denoted by $v_{i_1\dots i_g}\in\Lambda_gV$ with its dual basis $w^{i_1\cdots i_g}\in\Lambda^gV$, while the Leibniz symmetric basis will be $\partial_{i_1}^{\mu_1}\dots\partial_{i_g}^{\mu_g}\in L_gV$ with $\epsilon_{i_1}^{\mu_1}\dots\epsilon_{i_g}^{\mu_g}\in L^gV$ dual elements.
\end{definition}
Higher-order composite tensor elements will be denoted by an oriented-multi-set $X\in\text{OMSet}$ such that $w_X = \bigotimes_k w_{i_k}^{\otimes\mu_k}$ with $X = \plr{(i_1,\mu_1),\dots,(i_g,\mu_g)}$ and $|X|=\sum_k\mu_k$ is \verb`grade`+\verb`order`.
%There are two orientations and higher multiplicities result in zero for anti-symmetric tensors denoted by indices $\Lambda X\subseteq\Lambda V$, so the only interesting multiplicity is $\mu_k\equiv 1$.
Indices $\Lambda X\subseteq\Lambda V$ that are anti-symmetric have two orientations and higher multiplicities of those result in zero values, so the only interesting multiplicity is $\mu_k\equiv 1$.
Symmetric tensors have an ambiguous multiplicity of nilpotence; decide that $\epsilon_k^{\mu+1}=0$, so $\mu_k\leq\mu$ can be non-trivial, negative, or possibly unbounded.
Grassmann's exterior algebra is fundamentally simpler in structure than the symmetric generated algebra, as it doesn't invoke the properties of multi-sets.
The exterior Grassmann index algebra is related to the algebra of oriented sets and the Leibniz symmetric algebra is that of unoriented multi-sets. 
Combined, the mixed-symmetry algebra yield a multi-linear propositional lattice.
The formal sum of equal \verb`grade` elements is an oriented \verb`Chain` and with mixed \verb`grade` it is a \verb`MultiVector`.
Thus, various standard operations on the oriented multi-sets are possible including $\cup,\cap,\oplus$ and most importantly the $X\ominus Y=(X\cup Y)\bs(X\cap Y)$ symmetric difference operation $\veebar$.

In order to work with a \verb`TensorAlgebra{V}`, it is necessary for some computations to be cached. This is usually done automatically when accessed.
The staging of the precompilation and caching is designed so that a user can smoothly transition between very high dimensional and low dimensional algebras in a single session, with varying levels of extra caching and optimizations.
The parametric type formalism in \verb`Grassmann` is highly expressive to enable the pre-allocation of geometric algebra computations for specific sparse-subalgebras, including the representation of rotational groups.
It is possible to reach \verb`Basis` elements having up to $n=62$ indices, % with \verb`TensorAlgebra` having higher maximum dimensions than supported by Julia natively.
requiring full alpha-numeric labeling with lower-case and capital letters. 
However, Julia is only able to allocate full \verb`MultiVector` for $n\leq22$, with only sparse operations available at higher dimension form there.
While \verb`Grassmann.Algebra{V}` is a container for the \verb`TensorAlgebra` generators of $V$, the \verb`Grassmann.Algebra` is only cached for $n\leq8$.
For a \verb`VectorBundle{n}` of dimension $8<n\leq22$, the \verb`Grassmann.SparseAlgebra` type is used.
To reach higher dimensions, for $n>22$ the \verb`Grassmann.ExtendedAlgebra` type is used.

\section{Geometric algebraic product structure}

While oriented sets of the Grassmann exterior algebra are simpler, the parity \cite{artin} of $(-1)^\Pi$ is factored into transposition compositions when interchanging the ordering of the tensor product argument permutation.
Alternatively, symmetric algebra does not need to track the parity but does have multiplicity in its indices.
Symmetric differential function algebra of Leibniz trivializes the orientation property of index multi-sets, while Grassmann's exterior algebra is partitioned into two oriented equivalence classes by anti-symmetry.
The full tensor algebra can be sub-partitioned into equivalence classes in multiple ways based on the composite symmetry, grade, and metric signature properties.
%It can be said that the Leibniz symmetry class has a neutral orientation, while Grassmann's anti-symmetric class is partitinoned into positive and negative orientation.
Both of the symmetry classes can be characterized by the same geometric product, which is written as multiplication but explicitly denoted by $\ominus$ for clarity here.
%It is possible to express this universal product in terms of ordered-multi-set operations.

\begin{definition}
	The \textit{geometric algebraic product} is the $\Pi$ oriented symmetric difference operator $\ominus$ (weighted by the bilinear form $g$) and multi-set sum $\oplus$ applied to multilinear tensor products $\otimes$ in a single operation:
	$$ \omega_X\ominus \eta_Y = \ub{\text{$\Lambda^1$-anti-symmetric, $\Lambda^g$-mixed-symmetry}}{(-1)^{\Pi(\Lambda X,\Lambda Y)}\det\mat{g_{\Lambda(X\cap Y)}} (\bigotimes_{k\in \Lambda(X\ominus Y)} w^{i_k})}\otimes \ub{\text{$L^g$-symmetric}}{(\bigotimes_{k\in L(X\oplus Y)} \epsilon_{i_k}^{\otimes\mu_k})}$$
\end{definition}

\begin{note}
	The product symbol $\ominus$ will be used to expliclity denote usage of the geometric algebraic product, altough the standard number product \verb`*` notation could also be used. This choice is to help emphasize that the geometric algebraic product is characterized by symmetric differencing of anti-symmetric indices.
\end{note}

\begin{definition}
	The symmetry properties of the tensor algebra can be characterized in terms of the geometric product by two averaging operations, which are the symmetrization $\odot$ and anti-symmetrization $\boxtimes$ operators:
	\begin{align*}
		\bigodot_{k=1}^j \omega_k &= \frac1{j!}\sum_{\sigma\in S_P}\bigominus_k\omega_{\sigma(k)}, &
		\bigboxtimes_{k=1}^j \omega_k &= \frac1{j!}\sum_{\sigma\in S_P}(-1)^{\Pi(\sigma)}\bigominus_k\omega_{\sigma(k)}
	\end{align*}
\end{definition}
These products satisfy various \verb`MultiVector` properties, including the associative and distributive laws.

\begin{definition}[Exterior product]
	Let $w_k:V^{p_k}\times V'^{q_k}\ra\mb K$ such that $k\in\Lambda^{p_k}V$, then for all $\sigma\in S_{\sum p_k}$
	$$ (-1)^{\Pi(\sigma)}(\bigotimes_k \omega_k)(v_{\sigma(1)},\dots,v_{\sigma(\sum p_k)}) \sim \bigwedge_k \omega_k(v_{1},\dots,v_{p_k}) \iff \bigominus_k\omega_k = \bigboxtimes_k\omega_k $$
	there is an equivalence relation $\sim$ which holds. It has become typical to use the $\wedge$ product symbol to denote products of such elements as $\bigwedge\Lambda V \equiv \bigotimes\Lambda V/\sim$ modulo anti-symmetrization.
\end{definition}

\begin{definition}[Symmetric tensor of Leibniz differentials]
	Let $\partial_k = \frac\partial{\partial x_k}\in L_gV\,$ Leibnizian symmetric tensors, then there is an equivalence relation $\asymp$ which holds for each $\sigma\in S_p$ along with each derivation property,
	\begin{align*}
		(\bigotimes_k \partial_{\sigma(k)})\omega &\asymp (\partial_p \circ \dots\circ  \partial_1)\omega \iff \bigominus_k\partial_k = \bigodot_k\partial_k, & \partial_k(\omega\eta) &= \partial_k(\omega)\eta + \omega\partial_k(\eta).
	\end{align*}
\end{definition}

%\subsubsection{Interoperability for TensorAlgebra}

%\textit{AbstractTensors.jl} provides the abstract interoperability between tensor algebras having differing \verb`VectorBundle` parameters. The \verb`VectorBundle` unions and intersections are handled separately in a different package and the actual tensor implementations are handled separately also. This enables anyone who wishes to be interoperable with \verb`TensorAlgebra` to build their own subtypes in their own separate package with interoperability automatically possible between it all, provided the guidelines are followed.

%The key to making the whole interoperability work is that each \verb`TensorAlgebra` subtype shares a \verb`VectorBundle` parameter (with all \verb`isbitstype` parameters), which contains all the info needed at compile time to make decisions about conversions. So other packages need only use the vector space information to decide on how to convert based on the implementation of a type. If external methods are needed, they can be loaded by \verb`Requires` when making a separate package with \verb`TensorAlgebra` interoperability.

Since \verb`VectorBundle` choices are fundamental to \verb`TensorAlgebra` operations, the universal interoperability between \verb`TensorAlgebra{V}` elements with different associated \verb`VectorBundle` choices is naturally realized by applying the \verb`union` morphism to type operations.
For example, $\bigwedge :\Lambda^{p_1}V_1\times\dots\times\Lambda^{p_g}V_g \ra \Lambda^{\sum_kp_k}\bigcup_k V_k$. 
% where $\bigwedge \Lambda V \equiv \bigotimes \Lambda V/\sim$ is made interoperable by $\cup$.

\begin{definition}[Reverse, involute, conjugate]
	The \verb`reverse` of $\gen{\omega}_r$ is defined as $\gen{\tilde\omega}_r = (-1)^{(r-1)r/2}\gen{\omega}_r$, while the \verb`involute` is $(-1)^r\gen{\omega}_r$ and Clifford \verb`conj`  $\gen{\omega}_r^\ddagger$ is the composition of \verb`involute` and \verb`reverse`.
\end{definition}

\begin{definition}[Reversed product]
	The \textit{reversed geometric product} $\ast$ yields a Hilbert space structure:
	\begin{align*}
		\omega\ast\eta &= \tilde\omega\ominus\eta, & \omega\ast'\eta &= \omega\ominus\tilde\eta, & |\omega|^2 &= \omega\ast\omega, & |\omega| &= \sqrt{\omega\ast\omega}, & ||\omega|| = \text{Euclidean }|\omega|
	\end{align*}
\end{definition}
\begin{definition}[Sandwich product]
	The sandwich product is defined as $\eta\oslash\omega = (\eta\ast\omega)\ominus\eta$. Alternatively, the reversed definition is $\eta\obackslash\omega = \eta\ominus(\omega\ast'\eta)$ or in Julia $\eta\,$\verb`>>>`$\,\omega$, which is often found in literature.
\end{definition}
\begin{note}
	Observe that $\ast$ and $\ast'$ could both be used in \verb`abs`, \verb`abs2`, and \verb`norm`; however they are different products.
	The \textit{scalar product} $\circledast$ is the scalar part of the reversed product $\eta\circledast\omega = \gen{\eta\ast\omega}$.
	In general $\sqrt\omega = e^{(\log\omega)/2}$ is valid for \verb`inv`ertible multivectors $\omega\inv = \omega\ast|\omega|^{-2} = \tilde\omega/|\omega|^2$, where $\eta/\omega = \eta\ominus\omega\inv = \eta\ominus(\tilde\omega/(\omega\oslash1))$.
\end{note}
\begin{remark}
	It is overall more simple to use the $\ast$ and $\oslash$ operations instead of the $\ast'$ and $\obackslash$ variations.
\end{remark}
The \verb`real` part $\Re\omega = (\omega+\tilde\omega)/2$ is defined by the property $|\Re\omega|^2 = (\Re\omega)^{\ominus2}$ and the \verb`imag` part $\Im\omega = (\omega-\tilde\omega)/2$ by $|\Im\omega|^2 = -(\Im\omega)^{\ominus2}$, such that for any multivector $\omega = \Re\omega+\Im\omega$ has real and imaginary partitioned by 
$$\gen{\tilde\omega}_r/\mlr{\gen{\omega}_r} = \sqrt{\gen{\tilde\omega}_r^2/\big|\gen{\omega}_r\big|^2} = \sqrt{\gen{\omega}_r\ast\gen{\omega}_r\inv} = \sqrt{\gen{\tilde\omega}_r/\gen{\omega}_r}=\sqrt{(-1)^{(r-1)r/2}} \in\set{1,\sqrt{-1}},$$
which is a unique partitioning completely independent of the metric space and manifold of the algebra \cite{chappell-iqbal-gunn-abbott}.
$$ \omega\ast\omega = |\omega|^2 = |\Re\omega+\Im\omega|^2 = |\Re\omega|^2 + |\Im\omega|^2 + \Re\omega\ast\Im\omega + \Im\omega\ast\Re\omega = |\Re\omega|^2+|\Im\omega|^2 + 2\Re(\Re\omega\ast\Im\omega) $$
The \verb`radial` and \verb`angular` components in a multivector exponential are partitioned by the parity of their metric.

The Grassmann \verb`Basis` elements $v_k\in\Lambda_1V$ and $w^k\in\Lambda^1V$ are linearly independent vector and covector elements of $V$, while the Leibniz \verb`Operator` elements $\partial_k\in L_1V$ are partial tangent derivations and $\epsilon_k(x)\in L^1V$ are dependent functions of the \verb`tangent` manifold. 
Higher \verb`grade` elements of $\Lambda V$ correspond to \verb`SubManifold` subspaces, while higher \verb`order` function elements of $LV$ become homogenous polynomials and Taylor series.

%\subsection{Grassmann's duality via Hodge complement}

A universal unit volume element can be specified in terms of \verb`LinearAlgebra.UniformScaling`, which is independent of $V$ and has its interpretation only instantiated by the context of the \verb`TensorAlgebra{V}` element being operated on.
The universal interoperability of \verb`LinearAlgebra.UniformScaling` as a pseudoscalar element which takes on the \verb`VectorBundle` form of any other \verb`TensorAlgebra` element is handled globally. 
This enables the usage of \verb`I` from \verb`LinearAlgebra` as a universal pseudoscalar element defined at every point $x$ of the manifold, which is mathematically denoted by $I=I(x)$ and specified by the $g(x)$ bilinear tensor field of $TM$.

\begin{definition}[Grassmann-Poincare-Hodge complement]
	Let $\star\gen{\omega}_p = \gen{\omega}_p\ast I$, %= (-1)^{\frac{p(p+1)}2+\us k\sum i_k}\det\mat{ g_{i_ki_k}}\bigwedge_{m\in J(\omega)} w_m
	 then $\star : \Lambda^pV\ra\Lambda^{n-p}V$.
\end{definition}
\begin{remark}
	While $\star\omega$ is \verb`complementright` of $\omega$, the \verb`complementleft` would be $I\ast'\omega$. The $\star$ symbol was added to the Julia language as unary operator for ease of use with \verb`Grassmann` on Julia's v1.2 release.
\end{remark}

\section{Leibniz operators and Grassmann's Hodge-DeRahm theory}

John Browne has discussed the Grassmann duality principle in book \cite{browne}, stating that every theorem (involving exterior and regressive products) can be translated into its dual theorem by replacing the $\wedge$ and $\vee$ operations and applying \textit{Poincare duality} (homology).
First applying this Grassmann duality principle to the $\wedge$ product alone, let $\set{\omega_k}_k\in\Lambda^{p_k}V, P=\sum_kp_k$, then the co-product
$\bigvee :\Lambda^{p_1}V_1\times\dots\times\Lambda^{p_g}V_g \ra \Lambda^{P-(g-1)\#V}\bigcup_k V_k$ is obtained.
in Grassmann's original notation $\wedge,\vee,\star$ operations were combined.
The join $\wedge$ product is analogous to union $\cup$, the meet $\vee$ product is analogous to intersection $\cap$, and the orthogonal complement $\star\mapsto^\perp$ is negation.
Together, $(\wedge,\vee,\star)$ yield an orthocomplementary propositional lattice (quantum logic) by
\begin{align*}
	(\star\bigvee_k &\omega_k)(v_1,\dots,v_P) = (\bigwedge_k\star\omega_k)(v_1,\dots,v_P) \quad \textit{DeMorgan's Law}.
\end{align*}
\begin{definition}
	The left $\lrcorner$ and right $\llcorner$ contraction symmetrically define
	$ \gen{\omega}_r\cdot\gen{\eta}_s = \begin{cases} \omega\llcorner\eta=\omega\vee\star\eta & r\geq s \\ \omega\lrcorner\eta=\eta\vee\star\omega & r\leq s \end{cases} $.
	Note that for $\omega,\eta$ of equal grade, the operations $\omega\circledast\eta = \omega\odot\eta = \omega\cdot\eta = \omega\llcorner\eta = \omega\lrcorner\eta$ are symmetric.
\end{definition}

\begin{definition}
	Let $\nabla = \sum_k\partial_kv_k$ and $\epsilon = \sum_k\epsilon_k(x)w_k \in \Omega^1V$ be unit sums of the mixed-symmetry basis.
	Elements of $\Omega^pV$ are known as \textit{differential} $p$-\textit{forms} and both $\nabla$ and $\epsilon$ are \textit{tensor fields} dependent on $x\in W$.
	Another notation for a differential form is $dx_k = \epsilon_k(x)w_k$, such that $\epsilon_k = dx_k/w_k$ and $\partial_k\omega(x) = \omega'(x)$.
\end{definition}
\begin{note}
	The space $W$ does not have to equal $V\in\text{Vect}_{\mb K}$ above, as $\Omega^pV$ could have coefficients from $\mb K = LW$.
\end{note}
\begin{definition}
	Define \cite{bishop-goldberg} the differential $d:\Omega^p V\ra\Omega^{p+1}V$ and co-differential $\delta:\Omega^pV\ra\Omega^{p-1}V$ such that
	\begin{align*}
		\star d\omega &= \star(\nabla\wedge\omega) = \nabla\times\omega, & \omega\cdot\nabla &= \omega\vee\star\nabla = \partial\omega =-\delta\omega, 
	\end{align*}
	These two maps have the special properties $d\circ d=0$ and $\partial\circ\partial = 0$ for any form $\omega$ and vector field $\nabla$. 
	In topology there is \textit{boundary} operator $\partial$ that can be defined by $\partial\epsilon = \epsilon\cdot\nabla = \sum_k\partial_k\epsilon_k$ and is commonly discussed in terms the limit $\epsilon(x)\cdot\nabla\omega(x) = \lim_{h\ra0} \frac{\omega(x+h\epsilon)-\omega(x)}{h}$, which is the directional derivative \cite{sobczyk}.
\end{definition}
\begin{example}
	[Vorticity of vector-field]
	$\star d(dx_1+dx_2+dx_3) = (∂_2 -∂_3)dx_1 + (∂_3 -∂_1)dx_2 + (∂_1 -∂_2)dx_3$.
\end{example}
\begin{example}
	[Boundary of 3-simplex]
	Faces as oriented \verb`Chain`: $\partial(w_{1234}) = -\partial_4w_{123}+\partial_3w_{124}-\partial_2w_{134}+\partial_1w_{234}$.
\end{example}
\begin{theorem}[Leibniz-Taylor series]
	Let $\partial_X=\bigominus_k\partial_k^{\mu_k}$ be defined so that $|X|=\sum_k\mu_k$, then $e^{\partial\epsilon}\omega(x)$ is
	\begin{align*}
		\sum_{j=0}^\mu \frac{(\partial\epsilon)^{\ominus j}}{j!}\omega(x)
		= \sum_{j=0}^\mu\sum_{|X|=j} \bigominus_k \frac{(\partial_{k}\epsilon_k(x))^{\mu_{k}}}{\mu_k!}\omega(x)
		+ (\mu+1)\sum_{|X|=\mu+1}\int_0^1(1-t)^\mu\bigominus_k\frac{(\partial_k\epsilon_k(x))^{\mu_k}}{\mu_k!}\omega(x+t\epsilon)\,dt.
	\end{align*}
\end{theorem}
The multivariate \textit{chain rule} is encoded into the geometric algebraic product when using mixed-symmetry.

\begin{theorem}[Hilbert adjoint Hodge-DeRahm operators]
	Let $ \nabla \in\Omega_1 V $ be a Leibnizian vector field operator, then $d,-\partial$ are Hilbert adjoint Hodge-DeRahm operators with $\gen{\,\ast\,}$
	\begin{align*}
		\int_M d\omega\wedge\star\eta +\int_M \omega\wedge\star\partial\eta &= 0, & \gen{d\omega\ast\eta} &=\gen{\omega\ast-\partial\eta}
	\end{align*}
\end{theorem}
\begin{proof}
	Observe that $\partial\omega = \omega\cdot\nabla = \star(\star\omega\wedge\star^2\nabla) = (-1)^n(-1)^{nk}\star d\star\omega$.
	Then  substitute this into the integral $\int_M \omega\wedge(-1)^{mk+m+1}\star\star d\star\eta = (-1)^{km+m+1}(-1)^{(m-k+1)(k-1)}\int_M\omega\wedge d\star\eta$,
	with $(-1)^{km+m+1}(-1)^{(m-k+1)(k-1)}=(-1)^k$ and also
	$ (-1)^k\int_M\omega\wedge d\star\eta = \int_M d(\omega\wedge\star\eta) - (-1)^{k-1}\omega\wedge d\star\eta = \int_M d\omega\wedge\star\eta$.
	The other identity can be proved by relying on a variant of the \textit{common factor theorem} by Browne \cite{browne}.
\end{proof}
\begin{theorem}[Clifford-Dirac-Laplacian]
	Dirac operator $ \Delta^\frac12\omega = \mp\omega\ominus\nabla = \pm\nabla\wedge\omega \mp \omega\llcorner\nabla  = \pm d\omega\mp\partial\omega$ \cite{garling}.
	\begin{align*}
		\Delta\omega &= \nabla\wedge(\omega\cdot\nabla) + (\nabla\wedge\omega)\cdot\nabla) = \pm(\pm\omega\ominus\nabla)\ominus\nabla), & \mc H^p M = \setc{\Delta\omega = 0}{\omega\in \Omega^pM}
	\end{align*}
	Elements $\omega$ are \textit{harmonic} if $\Delta\omega = 0$ and both \textit{exact} $d\omega=0$ and \textit{coexact} $\delta\omega=0$, Hodge decomposition \cite{ivancevic}:
	$\Omega^pM=\mc H^pM\oplus\im{d\Omega^{p-1}M}\oplus\im{\delta\Omega^{p+1}M}$.
	Warning $\nabla\omega\neq\omega\nabla, \nabla^2\omega=\omega\nabla^2$ for higher-order tensor fields!
\end{theorem}

%\subsection{Projective null-cones of Dirac-Clifford-Hestenes}

\begin{definition}
	[Null-basis of projective split]
	Let $v_\pm^2 = \pm1$ be a basis with $v_\infty = v_++v_-$ and $v_\emptyset = (v_--v_+)/2$
\end{definition}
An embedding space $\mb R^{n+1,1}$ carrying the action from the group $SO(n+1,1)$ then has
$v_\infty^2 =0$, $v_\emptyset^2 =0$,
$v_\infty \cdot v_\emptyset = 1$,  and $v_{\infty\emptyset}^2 = 1$ with
Minkowski plane $v_{\infty\emptyset}$ having the Hestenes-Dirac-Clifford product properties
\begin{align*}
	v_{\infty\emptyset}\ominus v_\infty &= -v_\infty, &  v_{\infty\emptyset}\ominus v_\emptyset &= v_\emptyset, &
	v_\infty\ominus v_\emptyset &= -1 + v_{\infty\emptyset}, & v_\emptyset\ominus v_\infty &=  -1 - v_{\infty\emptyset}
\end{align*}
Declaring an additional \textit{null-basis} is done by specifying it in the string constructor with \verb`∞` at the first index (i.e. \verb`S"∞+++"`). Likewise, an optional \textit{origin} can be declared by \verb`∅` subsequently (i.e. \verb`S"∅+++"` or \verb`S"∞∅+++"`). These two basis elements will be interpreted in the type system such that they propagate under transformations when combining a mixed index sets (provided the \verb`Signature` is compatible).
\begin{example}[\textit{using Leibniz $\bigoplus\,$Grassmann}]
	\textbf{julia>} S"$\infty\emptyset\text{+++"}(\nabla)$\verb`^`$2 \,\,  \mapsto\, \, \Delta = (-2\partial_{\infty\emptyset} + \partial_1^2 + \partial_2^2 + \partial_3^2)\vec v $
\end{example}

\section{Conclusion}

\textit{Grassmann.jl} and its accompanying support packages provide an extensible platform for fully generalized computing with geometric algebra at high dimensions.
%This will enable the usage of many different types of \verb`TensorAlgebra` along with various \verb`VectorBundle` parameters and interoperability for a wide range of scientific and research applications.
All of the types and operations in this paper are implemented using only a few thousand lines of code with Julia's type polymorphism code generation, although the mixed-symmetry interaction of \verb`Leibniz` and \verb`Grassmann` is still a work in progress.
The goal is to algebraically define operator actions on groups of \textit{Leibniz.jl} differential function elements. 
One way is to use \textit{Reduce.jl} for differentiation and \textit{SyntaxTree.jl} to obtain optimal symbolic polynomial forms \cite{optim-poly}.

Thus, these packages provide general rotational algebras and Lie bivector groups with a full trigonometric suite enabled by $ e^{A} = \sum_n \frac{A^n}{n!} $.
Conformal geometric algebra is possible with the Minkowski plane $v_{\infty\emptyset}$, based on the null-basis.
In general, multivalued quantum logic is enabled by the $\wedge,\vee,\star$ Grassmann-Hodge dual lattice.
Mixed-symmetry algebra with \textit{Leibniz.jl} and \textit{Grassmann.jl}, having the geometric algebraic product chain rule, yields automatic differentiation and Hodge-DeRahm co/homology  as unveiled by Grassmann.
\begin{align*}
	0 &\os d{\us\partial\rightleftarrows} \Omega^0(M) \os d{\us\partial\rightleftarrows} \Omega^1(M) \os d{\us\partial\leftrightarrows} \cdots \os d{\us\partial\rightleftarrows} \Omega^n(M) \os d{\us\partial\rightleftarrows} 0, & \mc H^pM &\cong \frac{\ker(d\Omega^pM)}{\im{d\Omega^{p-1}M}}, & \dim\mc H^pM &= \dim\frac{\ker(\delta\Omega^pM)}{\im{\delta\Omega^{p+1}M}}
\end{align*}
In conclusion, the Dirac-Clifford product yields generalized Hodge-Laplacian and $b_p=\dim\mc H^pM$ are the Betti numbers with Euler characteristic $\chi = \sum_p (-1)^pb_p$.
There will be a more detailed follow-up paper.

\textbf{Acknowledgements}: Many thanks for discussions with Steven De Keninck, Utensil Song, Alexander Arsenovic, Hugo Hadfield, Karl Wessel; the helpful Julia community (e.g. Jameson Nash, Stefan Karpinski, Jeff Bezanson, Kristoffer Carlsson, Petr Krysl, Nathan Smith, etc); and thanks for support from my family.

%\cite{bezanson2017julia}

% **************GENERATED FILE, DO NOT EDIT**************

\bibliographystyle{juliacon}
\bibliography{ref.bib}


\end{document}
